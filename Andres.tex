%documento de clase
\documentclass[a4paper,12pt]{article}
%paquetes
\usepackage[spanish]{babel}
\usepackage[total={18cm, 21cm}, top=2cm, left=2cm]{geometry}
\usepackage{amsmath, amssymb, amsfonts, latexsym}
\usepackage[utf8]{inputenc}
\usepackage{graphicx}
 \usepackage{color}

%comandos
\parindent = 0mm

\author{Andrés Soledispa}

\title{Hola}

\date{\today}

%contenido
\begin{document}
\maketitle

Análisis

praparatoria
\section[EPN]{Escuela Politécnica Nacional}
\subsection{La historia de la patada mística}
\subsubsection{Comienzo}
\emph{\textbf{\textbackslash{\LARGE Topología}}}

\Large 
\color{red}
Cogito, ergo sum

\color{blue}
\LARGE
\centerline{Cogito, ergo sum}
\begin{center}

El día que la tierra se canso de girar,\hspace{10cm} ya no se que más escribir.
\vspace{10cm}Análisis Complejo. 

Probemos un poco del texto gris

%Para comentar

\noindent Las fuerzas fundamentales de la naturaleza son:
\begin{enumerate}
\item {La fuerza gravitatoria.}
\item {La fuerza electromagnética.} 
\end{enumerate}

\end{center}

\end{document}

